\documentclass[11pt,a4paper]{scrartcl}
\usepackage[latin9]{inputenc}
\usepackage{ucs}
\usepackage{amsmath}
\usepackage{amsfonts}
\usepackage{amssymb}
\author{Simon Gr�tzinger, MatNr.: 3600830\\Andr� Freitag, MatNr.: 3601865\\Ralf Vogler, MatNr.: 3602420}
\title{Machine Learning Worksheet 2}

\makeatletter
\renewcommand*\env@matrix[1][*\c@MaxMatrixCols c] 
    }

\begin{document}
\maketitle
\section{Problem 1}
\textit{T .. Terrorist, S .. Scan positive}\\
\begin{align}
P(S | T) = P(\neg{S} | \neg{T}) = 0.95\\
P(T) = \frac{1}{100} = 0.01\\
P(S) = P(S | T) * P(T) + P(S | \neg{T}) * P(\neg{T}) = 0.95 * 0.01 + 0.05 * 0.99 = 0.059\\
P(T | S) = \frac{P(S | T) * P(T)}{P(S)} = \frac{ 0.95 * 0.01}{0.059} = 0.1610169492
\end{align}

\section{Problem 2}
\textit{$ B_1/B_2 $.. Ball 1/2 in box is red, $ D_1/D_2/D_3 $.. Drawn ball 1/2/3 is red}
\begin{align}
P(B_1) = P(B_2) = 0.5\\
P(B_1, B_2) = P(B_1) * P(B_2) = 0.25\\
P(B_1, B_2 | D_1, D_2, D_3) = \frac{P(D_1, D_2, D_3 | B_1, B_2) * P(B_1, B_2)}{P(D_1, D_2, D_3)}
= \frac{1 * 0.25}{P(D_1, D_2, D_3)}\\
P(D_1, D_2, D_3) = P(D_1, D_2, D_3 | B_1, B2) * P(B_1, B_2)\\
 + P(D_1, D_2, D_3 | \neg{B_1}, B2) * P(\neg{B_1}, B_2)\\
 + P(D_1, D_2, D_3 | B_1, \neg{B2}) * P(B_1, \neg{B_2})\\
 + P(D_1, D_2, D_3 | \neg{B_1}, \neg{B2}) * P(\neg{B_1}, \neg{B_2})\\
 = 0.25 * (1 + 2 * 0.5^3 + 0) = 0.3125\\
 P(B_1, B_2 | D_1, D_2, D_3) = \frac{0.25}{0.3125} = 0.8
\end{align}

\begin{align}
Ax &= \lambda x\\
Ax - \lambda x &= 0\\
(A-\lambda)x &= 0
\end{align}

for $\lambda_1 = 2$:
\begin{align}
\Rightarrow
\begin{pmatrix}[ccc|c]
0 & 1 & 0 & 0\\
1 & 0 & 1 & 0\\
0 & 1 & 0 & 0\\
\end{pmatrix}
\Rightarrow \dots \Rightarrow E_1 =
\underline{
\langle
\begin{pmatrix}
1\\0\\-1
\end{pmatrix}
\rangle
}
\end{align}

for $\lambda_2 = 2+\sqrt{2}$:
\begin{align}
\Rightarrow
\begin{pmatrix}[ccc|c]
\sqrt{2} & 1 & 0 & 0\\
1 & \sqrt{2} & 1 & 0\\
0 & 1 & \sqrt{2} & 0\\
\end{pmatrix}
\Rightarrow \dots \Rightarrow E_2 =
\underline{
\langle
\begin{pmatrix}
1\\ -\sqrt{2} \\1
\end{pmatrix}
}
\rangle
\end{align}

for $\lambda_3 = 2-\sqrt{2}$:
\begin{align}
\Rightarrow
\begin{pmatrix}[ccc|c]
-\sqrt{2} & 1 & 0 & 0\\
1 & -\sqrt{2} & 1 & 0\\
0 & 1 & -\sqrt{2} & 0\\
\end{pmatrix}
\Rightarrow \dots \Rightarrow E_3 =
\underline{
\langle
\begin{pmatrix}
1\\ \sqrt{2} \\1
\end{pmatrix}
}
\rangle
\end{align}

\subsection{Octave code}
\begin{lstlisting} 
	%start
	A = [2 -1 0; -1 2 -1; 0 -1 2]
	%[v,d] = eig(A)
	e = roots(poly(A))
	e(1) % eigenvalue 1
	e(2) % eigenvalue 2
	e(3) % eigenvalue 3
	v1 = null((eye(3)*e(1))-A) % eigenvector 1
	v2 = null((eye(3)*e(2))-A) % eigenvector 2
	v3 = null((eye(3)*e(3))-A) % eigenvector 3
\end{lstlisting}

\section{Problem 2.}


\begin{align}
B &\in \mathbb{R}^{n\times n}\\
U &= \begin{pmatrix}
| &| &|\\
x_1 &\dots &x_n\\
| &| &|\\
\end{pmatrix}\\
D &= \begin{pmatrix}
\lambda_1 & &\\
& \ddots &\\
& & \lambda_n\\
\end{pmatrix}\\
\end{align}

\begin{align}
B &= UDU^{-1}\\
BU &= UD\\
\begin{pmatrix}
$---$ &b_1^{T} &$---$\\
$---$ &\dots &$---$\\
$---$ &b_n^{T} &$---$
\end{pmatrix}
\begin{pmatrix}
| &| &|\\
x_1 &\dots &x_n\\
| &| &|\\
\end{pmatrix}
&=\begin{pmatrix}
| &| &|\\
x_1 &\dots &x_n\\
| &| &|\\
\end{pmatrix}
\begin{pmatrix}
\lambda_1 & &\\
& \ddots &\\
& & \lambda_n\\
\end{pmatrix}\\
\begin{pmatrix}
b_1^{T}x_1 &\dots &b_n^{T}x_n\\
\vdots &\dots &\vdots\\
b_1^{T}x_1 &\dots &b_n^{T}x_n
\end{pmatrix}
&= \begin{pmatrix}
| &| &|\\
x_1\lambda_1 &\dots &x_n\lambda_n\\
| &| &|\\
\end{pmatrix}
\end{align}

as $\forall i: Bx_i = \lambda x_1 = x_i\lambda \;\;\;\Rightarrow\;\;\;$ \underline{equation (22) is valid}~~~~~$\Box$

\section{Problem 3.}

\subsection{B has real eigenvalues}
\begin{align}
Bx &= \lambda x\\
\overline{x}^{T}Bx &= \overline{x}^{T}\lambda x
\end{align}
\begin{align}
Bx &= \lambda x\\
\overline{Bx} &= \overline{\lambda x}\\
B\overline{x} &= \overline{\lambda} \overline{x}\\
(B\overline{x})^{T} &= (\overline{\lambda} \overline{x})^{T}\\
\overline{x}^{T} B^{T} &= \overline{x}^{T} \overline{\lambda}^{T}\\
\overline{x}^{T} B &= \overline{x}^{T} \overline{\lambda}\\
\overline{x}^{T} B x &= \overline{x}^{T} \overline{\lambda} x
\end{align}

\begin{align}
(25) + (32) ~~\Rightarrow~~ \overline{x}^{T} \overline{\lambda} x &= \overline{x}^{T}\lambda x\\
\overline{\lambda} &= \lambda~~~~~~~~~~~~~~~~~~~~~~~~~\Box
\end{align}



\subsection{eigenvecotrs are orhtogonal if eigenvalues pairwise distinct}


with $\lambda \neq \mu: Bx_1 = \lambda x_1$ and $Bx_2 = \mu x_2$
\begin{align}
Bx_1 &= \lambda x_1\\
x_2^{T} Bx_1 &= x_2^{T}\lambda x_1
\end{align}

\begin{align}
Bx_2 &= \mu x_2\\
x_2^{T}B &= x_2^{T} \mu\\
x_2^{T}B x_1 &= x_2^{T} \mu x_1
\end{align}

\begin{align}
(36)+(39) \Rightarrow x_2^{T} \lambda x_1 &= x_2^{T} \mu x_1\\
\lambda x_2^{T} x_1 &= \mu x_2^{T} x_1\\
\text{as}~\lambda \neq \mu \Rightarrow x_2^{T} x1 &= 0~~~~~~~~~~~~~~~~~~~~~~\Box
\end{align}

\section{Problem 4.}

\begin{itemize}
\item $B$ is real and symmetric $\land$ $B$ has n distinct eigenvalues $\lambda_1,\hdots,\lambda_2$
\item $\Rightarrow$ the $n$ eigenvectors are orthogonal (3.2)
\item $\Rightarrow$ the $n$ eigenvectors are linear independent
\item $\Rightarrow$ $B$ is similar to a diagonal matrix sharing determinant and trace
\item $\Rightarrow$ $D_B = \begin{pmatrix}
\lambda_1 & &\\
& \ddots &\\
& & \lambda_n\\
\end{pmatrix}$ with:
\begin{itemize}
\item $det(D_B) = \prod \limits_{i=1}^{n} \lambda_i$
\item $tr(D_B) = \sum \limits_{i=1}^n \lambda_i$
\end{itemize}
\end{itemize}
~~~~~~~~~~~~~~~~~~~~~~~~~~~~~~~~~~~~~~~~~~~~~~~~~~~~~~~~~~~~~~~~~~~~~~~~~~~~~~~~~~~~$\Box$

\section{Problem 5.}

\subsection{}
\begin{align}
h(x_0) = \omega_0 + \omega^{T} x_0 &= 0\\
\omega^{T} x_0 &= -\omega_0
\end{align}

\subsection{}
\begin{align}
\omega^{T}x_1 = -\omega_0 &= \omega^{T} x_2\\
\Rightarrow \omega^{T} x_1 - \omega^{T} x_2 &= 0\\
\omega^{T}(x_1-x_2) &= 0
\end{align}

\subsection{}
\begin{align}
h(x) \cdot \hat{\omega} &= 0\\
(\omega_0 + \omega^{T}x)(\frac{\omega}{||\omega||}) &= 0\\
(\omega_0 - \omega_0) (\frac{\omega}{||\omega||}) &= 0\\
0 &= 0~~~~~~~~~~~~~~~~\Box
\end{align}

\subsection{}
\begin{align}
\hat{\omega}^T(x-x_0) &= (\frac{\omega}{||\omega||})^T(x-x_0)\\
&= \frac{1}{||\omega||} \omega^T (x-x_0)\\
&= (\omega^T x - \omega^T x_0) \frac{1}{||\omega||}\\
&= (\omega^T x - (-\omega_0)) \frac{1}{||\omega||}\\
&= (\omega^T x + \omega_0) \frac{1}{||\omega||}~~~~~\Box
\end{align}
\end{document}