\documentclass[11pt,a4paper]{scrartcl}
\usepackage[latin9]{inputenc}
\usepackage{ucs}
\usepackage{amsmath}
\usepackage{amsfonts}
\usepackage{amssymb}
\author{Simon Gr�tzinger, MatNr.: 3600830\\Andr� Freitag, MatNr.: 3601865\\Ralf Vogler, MatNr.: 3602420}
\title{Machine Learning Worksheet 2}

\makeatletter
\renewcommand*\env@matrix[1][*\c@MaxMatrixCols c] 
    }

\begin{document}
\maketitle
\section{Problem 1}
\textit{T .. Terrorist, S .. Scan positive}\\
\begin{align}
P(S | T) = P(\neg{S} | \neg{T}) = 0.95\\
P(T) = \frac{1}{100} = 0.01\\
P(S) = P(S | T) \cdot P(T) + P(S | \neg{T}) \cdot P(\neg{T}) = 0.95 \cdot 0.01 + 0.05 \cdot 0.99 = 0.059\\
P(T | S) = \frac{P(S | T) \cdot P(T)}{P(S)} = \frac{ 0.95 \cdot 0.01}{0.059} = 0.1610169492
\end{align}
\color{red}
$P(T) = \frac{1}{99}$ bzw. $P(\neg T) = \frac{98}{99}$ weil ich wei�, dass ich kein terrorist bin?\\
\color{black}

\section{Problem 2}
\textit{$ B_1/B_2 $.. Ball 1/2 in box is red, $ D_1/D_2/D_3 $.. Drawn ball 1/2/3 is red}
\begin{align}
P(B_1) = P(B_2) = 0.5\\
P(B_1, B_2) = P(B_1) \cdot P(B_2) = 0.25\\
P(B_1, B_2 | D_1, D_2, D_3) = \frac{P(D_1, D_2, D_3 | B_1, B_2) \cdot P(B_1, B_2)}{P(D_1, D_2, D_3)}
= \frac{1 \cdot 0.25}{P(D_1, D_2, D_3)}\\
P(D_1, D_2, D_3) = P(D_1, D_2, D_3 | B_1, B2) \cdot P(B_1, B_2)\\
 + P(D_1, D_2, D_3 | \neg{B_1}, B2) \cdot P(\neg{B_1}, B_2)\\
 + P(D_1, D_2, D_3 | B_1, \neg{B2}) \cdot P(B_1, \neg{B_2})\\
 + P(D_1, D_2, D_3 | \neg{B_1}, \neg{B2}) \cdot P(\neg{B_1}, \neg{B_2})\\
 = 0.25 \cdot (1 + 2 \cdot 0.5^3 + 0) = 0.3125\\
 P(B_1, B_2 | D_1, D_2, D_3) = \frac{0.25}{0.3125} = 0.8
\end{align}

\color{red}
$X := \#$(red balls in box), $Y := \#$(red draws)
\begin{align}
P(X=2 ~|~ Y=3) &= \frac{P(Y=3 ~|~ X=2)\cdot P(X=2)}{P(Y=3)}\\
&= \frac{1 \cdot \frac{1}{4}}{\sum \limits_{i=0}^2 P(Y=3 ~|~ X=i)\cdot P(X=i)}\\
&= \frac{\frac{1}{4}}{0 + \frac{1}{2}^3\cdot \frac{1}{2} + \frac{1}{4}} = \frac{4}{5}
\end{align}
\color{black}

\section{Problem 3.}

\begin{align}
E[X] &= \int_{-\infty}^{\infty} x \cdot p(x) dx\\
&= \int_{-\infty}^{0} x \cdot p(x) dx + \int_{0}^{1} x \cdot p(x) dx + \int_{1}^{\infty} x \cdot p(x) dx\\
&= 0 + [\frac{1}{2}x^2]_{0}^1 + 0\\
&= \frac{1}{2}
\end{align}

\section{Problem 4.}
\color{red}?\color{black}


\section{Problem 5.}

\begin{align}
c \cdot P(X > c) &\leq c \cdot \sum \limits_{x\in \Omega_X :~x>c} x\cdot P(X=x)\\
&\leq \sum \limits_{x\in \Omega_X :~x>c} x\cdot P(X=x)\\
&\leq \sum \limits_{x\in \Omega_X} x\cdot P(X=x)\\
&= E[X]~~~~~~~~~~~~~~~~~~~~~~~~~~~~~~~~~~~~~\Box
\end{align}

$X := \#heads$ with $X \sim Bin(n,\frac{1}{2}) \Rightarrow E[X] = \frac{1}{2}n$
\begin{align}
P(X>\frac{3}{4}n) &\leq \frac{\frac{1}{2}n}{\frac{3}{4}n}\\
&= \frac{2}{3}
\end{align}

\section{Problem 6.}
\begin{align}
P(|X-E[X]| > a) &= P((X-E[X])^2 > a^2)\\
&\leq \frac{E[(X-E[X])^2]}{a^2}\\
&= \frac{Var[X]}{a^2}~~~~~~~~~~~~~~~~~~~~~~~\Box
\end{align}

$X := \#heads$ with $X \sim Bin(n,\frac{1}{2}) \Rightarrow Var[X] = n\cdot p \cdot q = \frac{1}{4}n$ and $E[X] = \frac{1}{2}n$
\begin{align}
P(|X-\frac{1}{2}n| > \frac{3}{4}n - \frac{1}{2}n) &\leq \frac{\frac{1}{4}n}{\frac{1}{16}n^2}\\
&= \frac{4}{n}
\end{align}

\section{Problem 7.}
$n=1$:
\begin{align}
f(\lambda_1x_1) \leq \lambda_1f(x_1)
\end{align}
$n=2$:
\begin{align}
f(\lambda_1x_1 + \lambda_2x_2) &= f(\lambda_1x_1 + (1-\lambda_1)x_2)\\
&= \lambda_1f(x_1) + (1-\lambda_1)f(x_2) ~~~~~~\text{(def. of convex function)}\\
&= \lambda_1f(x_1) + \lambda_2f(x_2)
\end{align}
$n=n+1$:
\begin{align}
f(\lambda_1x_1 + \hdots + \lambda_{n+1}x_{n+1}) &= f((1-\lambda_{n+1}) \sum \limits_{i=1}^{n}\frac{\lambda_i x_i}{1-\lambda_{n+1}} + \lambda_{n+1}x_{n+1})\\
&\leq \lambda_{n+1}f(x_{n+1}) + (1-\lambda_{n+1}) f\left(\sum \limits_{i=1}^{n}\frac{\lambda_i}{1-\lambda_{n+1}} x_i\right)\\
\text{with} ~~\sum \limits_{i=1}^{n}\frac{\lambda_i}{1-\lambda_{n+1}} = 1:\\
&\overset{I.B.}\leq \sum \limits_{i=1}^{n+1} \lambda_i f(x_i)~~~~~~~~~~~~~~~~~~~~~~~~~~~~~~~~~~~~~~~\Box
\end{align}


\end{document}