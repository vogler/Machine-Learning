\documentclass[11pt,a4paper]{scrartcl}
\usepackage[latin9]{inputenc}
\usepackage{ucs}
\usepackage{amsmath}
\usepackage{amsfonts}
\usepackage{amssymb}
\usepackage{graphicx}

\author{Simon Gr�tzinger, MatNr.: 3600830\\Ralf Vogler, MatNr.: 3602420}
\title{Machine Learning Worksheet 7}

\makeatletter
\renewcommand*\env@matrix[1][*\c@MaxMatrixCols c] 
    }

\begin{document}
\maketitle
\section{Problem 1 - (strictly) convex}
An interval of a function is called convex if its value is always less or equal than the value of the line between any two points on its graph.
It is called strictly convex if its value is always less.

For any two points $x_{1}$ and $x_{2}$ and any $t \in [0,1]$:
\paragraph*{Convex function}
\begin{align}
f(t x_{1} + (1-t) x_{2}) \leq t f(x_{1}) + (1-t) f(x_{2}).
\end{align}
Alternatively: $f''(x) \geq 0 \Leftrightarrow $ convex
\paragraph*{Strictly convex function}
\begin{align}
f(t x_{1} + (1-t) x_{2}) < t f(x_{1}) + (1-t) f(x_{2}).
\end{align}
Alternatively: $f''(x) > 0 \Rightarrow $ strictly convex


\section{Problem 2 - extrema}
Convex functions: a local minimum is also a global minimum.\\
A strictly convex function has at most one local minimum, which is also a global minimum.


\section{Problem 3 - examples}
Strictly convex: $x^{4}$\\
Convex but not strictly convex: $|x|$\\
\includegraphics[width=\textwidth]{wolframalpha-20111221072253038.png}


\section{Problem 5 - example}
\includegraphics[width=\textwidth]{plot.png}
The margin is $\sqrt{2}$. The samples (1,1), (2,2) and (1,3) are support vectors.
b is 3. \textbf{w} is $\frac{2}{\sqrt{2}}$.


\section{Problem 7 - linear separator}
The data can not be linearly separated because the convex hulls of the two sets overlap (-1.5 would be a linear separator if 3 was not in the set for label +1).

\section{Problem 8 - feature map}
\textbf{Label +1:} (-3, 9), (-2, 4), (3, 9)\\
\textbf{Label -1:} (-1, 1), (0, 0), (1, 1)

Yes, it is now separable. See plot:\\
\includegraphics[width=\textwidth]{plot2.png}


\section{Problem 9 - feature map}
\includegraphics[width=\textwidth]{plot3.png}
Decision boundary in the original feature space:\\
\includegraphics[width=\textwidth]{plot4.png}
Yes, it is possible. Adding points outside of the margin doesn't change the hyperplane.
\end{document}